\documentclass[a4paper, 12pt]{article}
\usepackage[utf8]{inputenc}
\usepackage[T1, T2A]{fontenc}
\usepackage[english, ukrainian]{babel}

\usepackage{amsmath, amssymb}
\usepackage{multicol}
\usepackage{graphicx}
\usepackage{float}
\usepackage{minted,verbatim}

\allowdisplaybreaks
\setlength\parindent{0pt}
\numberwithin{equation}{subsection}

\usepackage{hyperref}
\hypersetup{unicode=true,colorlinks=true,linktoc=all,linkcolor=blue}

\newcommand{\Sum}{\displaystyle\sum\limits}

\newcommand{\ZZ}{\mathbb{Z}}

\numberwithin{equation}{section}

\begin{document}

\begin{center}
\hfill \break
	Міністерство освіти та науки України \\
	Київський національний університет імені Тараса Шевченка \\ 
	Факультет комп'ютерних наук та кібернетики \\
	Кафедра обчислювальної математики
\end{center}

\vfill 

\begin{center}
	\large{
		Звіт до лабораторної роботи №3 на тему: \\ 
		``Мурашиний алгоритм''
	}
\end{center}

\vfill 

\begin{flushright}
	Виконав студент групи ОМ-3 \\
	Скибицький Нікіта
\end{flushright}

\vfill 

\begin{center}
    Київ, 2019
\end{center}

\thispagestyle{empty} 

\newpage

\tableofcontents

\section{Постановка задачі}

Розглянемо відому задачу про ранець: є $T$ типів предметів, причому предметів $i$-го типу рівно $q_i$ штук. Предмет $i$-го типу характеризується значеннями своєї корисності $u_i$ та ваги $w_i$. Окрім цього є ранець який вміщує довільну кількість предметів сумарною вагою не більше $W$. Необхідно вибрати підмножину заданих предметів яку можна розмістити у ранці, з максимальною сумарною корисністю. \medskip

Тобто, ставиться задача
\begin{equation}
	U(c) = \Sum_{i = 1}^T c_i \cdot u_i \xrightarrow[c \in \mathcal{C}]{} \max,
\end{equation}
де $\mathcal{C}$ --- допустима область, яка визначається наступним чином:
\begin{equation}
	\mathcal{C} = \left\{ c \in \ZZ^T \middle| \forall i: 0 \le c_i \le q_i \land \Sum_{i = 1}^T c_i \cdot w_i \le W \right\}.
\end{equation}

Зауважимо, що можна також записати
\begin{equation}
 	U(c) = \langle c, u \rangle,
\end{equation}
і
\begin{equation}
	\mathcal{C} = \left\{ c \in Q_1 \times Q_2 \times \ldots \times Q_T \middle| \langle c, w \rangle \le W \right\},
\end{equation}
де $Q_i = \ZZ \cap [0, q_i]$.

\section{Опис алгоритму}

Розглянемо популяцію з $N$ мурах, які протягом $M$ ітерацій намагаються спакувати свій ``ранець'' (наприклад, підготувати запаси на зиму, які необхідно розмістити в обмеженому просторі мурашника). Уявімо, що на кожній ітерації кожна мурашка пакує свій уявний ранець ходячи туди-сюди до потрібних їй предметів, залишаючи на своєму шляху феромени, і керуючись у вже наявними із попередніх ітерацій фероменами для вибору шляху.

\subsection{Феромени які залишає мурашка}

А саме, нехай муршака знайшла розв'язок $c$ з сумарною вагою $w$ і сумарною корисністю $u$, тоді кількість фероменів $f_i$ залишених на шляху до купки об'єктів типу $i$ описується функцією $f$:
\begin{equation}
	f_i = f(c_i, w, u, w_i, u_i),
\end{equation}
на яку накладаються наступні умови:
\begin{align}
	\frac{\partial f}{\partial w} &< 0, \\
	\frac{\partial f}{\partial u} &> 0, \\
	\frac{\partial f}{\partial w_i} &< 0, \\
	\frac{\partial f}{\partial u_i} &> 0, \\
	\frac{\partial f}{\partial c_i} &> 0, \\
	f(0, \ldots) &= 0.
\end{align}

Наприклад, 
\begin{equation}
	f(w, u, w_i, u_i) = \ln (c_i + 1) \cdot \frac{u_i}{w_i} \cdot \frac{u}{w}.
\end{equation}

\subsection{Процес побудови розв'язку}

Нехай з попередньої ітерації вже наявні феромени у кількості $f_i$ на шляху до купки предметів $i$-го типу, тоді поки у ранець можна покласти ще хоча б один предмет мурашка обчислює наступні характеристики $\chi_i$ кожного типу предметів який ще може влізти у ранець:
\begin{equation}
	\chi_i = \chi(u_i, w_i, f_i),
\end{equation}
де на функцію $\chi$ накладаються наступні умови
\begin{align}
	\frac{\partial \chi}{\partial w_i} &< 0, \\
	\frac{\partial \chi}{\partial u_i} &> 0, \\
	\frac{\partial \chi}{\partial f_i} &> 0.
\end{align}

Наприклад,
\begin{equation}
	\chi(w_i, u_i, f_i) = \frac{u_i}{w_i} \cdot (1 + \rho \cdot f_i),
\end{equation}
де $\rho \in (0, 2)$ --- певний коефіцієнт, наприклад $\rho = 1$. \medskip

На основі обчислених характеристик обчислюються ймовірності вибору кожного типу предметів:
\begin{equation}
	p_i = \frac{\chi_i}{\Sum_{i = 1}^T \chi_i}.
\end{equation}

\subsection{Процес оновлення фероменів}

Якщо до ітерації $i$ кількість фероменів на шляху до $j$-ої купки предметів дорівнює $F_{i,j}$, а на $i$-ій ітерації додалося $f_{i,j}$, то
\begin{equation}
	F_{i+1,j} = \alpha \cdot f_{i, j} + (1 - \alpha) \cdot F_{i, j},
\end{equation}
де $\alpha \in (0, 1)$ --- певний коефіцієнт, наприклад $\alpha = 0.1$.

\section{Код}

\subsection{Процес побудови розв'язку}

\inputminted{python}{generate_solution.py}

\subsection{Процес обчислення фероменів}

\inputminted{python}{calculate_feroments.py}

\subsection{Програма-драйвер і основний алгоритм}

\inputminted{python}{knapsack.py}

\section{Результати}

Проводилося тестування на відносно складній задачі ($T = 100$) але однорідній задачі. Однорідність означає, що
\begin{equation}
	\frac{\max_i u_i}{\min u_i} \le 2 \quad \land \quad \frac{\max_i w_i}{\min w_i} \le 2
\end{equation}

Початковий результат отриманий випадковим чином складав 80\% теоретичного максимуму:
\begin{verbatim}
it #00: best_u = 11882
\end{verbatim}

За 10 ітерацій алгоритм досяг покращення у 17\% у порівнянні з початковим результатом, тобто досяг результату у 97\% від теоретичного максимуму:
\begin{verbatim}
it #10: best_u = 13942
\end{verbatim}

Опісля відбувається стагнація: за наступні 100 ітерацій прогрес склав ще 2\%, у результаті чого алгоритм підійшов до 99\% теоретичного максимуму.
\begin{verbatim}
it #99: best_u = 14135
\end{verbatim}


\end{document}