\documentclass[a4paper, 12pt]{article}
\usepackage[T2A,T1]{fontenc}
\usepackage[utf8]{inputenc}
\usepackage[english, ukrainian]{babel}
\usepackage{amsmath}
\usepackage{amssymb}
\usepackage{mathtools}
\usepackage{euler}
% \usepackage[dvipsnames]{xcolor}
\usepackage{xcolor}
\usepackage[margin=1.5cm]{geometry}
\usepackage{float}
\usepackage{multirow}
\usepackage{multicol}
\usepackage{url}
\usepackage[unicode=true, colorlinks=true, linktoc=all, linkcolor=blue]{hyperref}
\usepackage{cite}
\usepackage{amsthm}
\usepackage{thmtools}
\usepackage[framemethod=TikZ]{mdframed}
\usepackage[toc,page,title,titletoc]{appendix}
\usepackage{bookmark}
\usepackage[nottoc,notlot,notlof]{tocbibind}
\usepackage{minted}
% \usepackage[hang]{footmisc}
\theoremstyle{definition}
\mdfdefinestyle{mdbluebox}{%
	roundcorner = 10pt,
	linewidth=1pt,
	skipabove=12pt,
	innerbottommargin=9pt,
	skipbelow=2pt,
	nobreak=true,
	linecolor=blue,
	backgroundcolor=TealBlue!5,
}
\declaretheoremstyle[
	headfont=\sffamily\bfseries\color{MidnightBlue},
	mdframed={style=mdbluebox},
	headpunct={\\[3pt]},
	postheadspace={0pt}
]{thmbluebox}

\mdfdefinestyle{mdredbox}{%
	linewidth=0.5pt,
	skipabove=12pt,
	frametitleaboveskip=5pt,
	frametitlebelowskip=0pt,
	skipbelow=2pt,
	frametitlefont=\bfseries,
	innertopmargin=4pt,
	innerbottommargin=8pt,
	nobreak=true,
	linecolor=RawSienna,
	backgroundcolor=Salmon!5,
}
\declaretheoremstyle[
	headfont=\bfseries\color{RawSienna},
	mdframed={style=mdredbox},
	headpunct={\\[3pt]},
	postheadspace={0pt},
]{thmredbox}

\declaretheorem[style=thmbluebox,name=Теорема,numberwithin=section]{theorem}
\declaretheorem[style=thmbluebox,name=Теорема,numbered=no]{theorem*}
\declaretheorem[style=thmbluebox,name=Лема,sibling=theorem]{lemma}
\declaretheorem[style=thmbluebox,name=Лема,numbered=no]{lemma*}
\declaretheorem[style=thmbluebox,name=Твердження,sibling=theorem]{proposition}
\declaretheorem[style=thmbluebox,name=Наслідок,sibling=theorem]{corollary}
\declaretheorem[style=thmredbox,name=Приклад,sibling=theorem]{example}

\mdfdefinestyle{mdgreenbox}{%
	skipabove=8pt,
	linewidth=2pt,
	rightline=false,
	leftline=true,
	topline=false,
	bottomline=false,
	linecolor=ForestGreen,
	backgroundcolor=ForestGreen!5,
}
\declaretheoremstyle[
	headfont=\bfseries\sffamily\color{ForestGreen!70!black},
	bodyfont=\normalfont,
	spaceabove=2pt,
	spacebelow=1pt,
	mdframed={style=mdgreenbox},
	headpunct={ --- },
]{thmgreenbox}

\mdfdefinestyle{mdblackbox}{%
	skipabove=8pt,
	linewidth=3pt,
	rightline=false,
	leftline=true,
	topline=false,
	bottomline=false,
	linecolor=black,
	backgroundcolor=RedViolet!5!gray!5,
}
\declaretheoremstyle[
	headfont=\bfseries,
	bodyfont=\normalfont\small,
	spaceabove=0pt,
	spacebelow=0pt,
	mdframed={style=mdblackbox}
]{thmblackbox}

% \theoremstyle{theorem}
\declaretheorem[name=Запитання,sibling=theorem,style=thmblackbox]{ques}
\declaretheorem[name=Вправа,sibling=theorem,style=thmblackbox]{exercise}
\declaretheorem[name=Зауваження,sibling=theorem,style=thmgreenbox]{remark}
\declaretheorem[name=Припущення,sibling=theorem,style=thmblackbox]{assumption}

\theoremstyle{definition}
\newtheorem{claim}[theorem]{Твердження}
\newtheorem{definition}[theorem]{Визначення}
\newtheorem{fact}[theorem]{Факт}

\newtheorem{problem}{Задача}[section]
\newtheorem{sproblem}[problem]{Задача}
\newtheorem{dproblem}[problem]{Задача}
\renewcommand{\thesproblem}{\theproblem$^{\star}$}
\renewcommand{\thedproblem}{\theproblem$^{\dagger}$}

\makeatletter
\newenvironment{solution}[1][\solutionname]{\par
  \pushQED{\qed}%
  \normalfont \topsep6\p@\@plus6\p@\relax
  \trivlist
%<amsbook|amsproc>  \itemindent\normalparindent
  \item[\hskip\labelsep
%<amsbook|amsproc>        \scshape
%<amsart|amsthm>        \itshape
\itshape 
    #1\@addpunct{.}]\ignorespaces
}{%
  \popQED\endtrivlist\@endpefalse
}
%    \end{macrocode}
%    Default for \cn{proofname}:
%    \begin{macrocode}
\providecommand{\solutionname}{Розв'язок}

\makeatother
\renewcommand{\phi}{\varphi}
\renewcommand{\epsilon}{\varepsilon}

\newcommand{\NN}{\mathbb{N}}
\newcommand{\ZZ}{\mathbb{Z}}
\newcommand{\QQ}{\mathbb{Q}}
\newcommand{\RR}{\mathbb{R}}
\newcommand{\CC}{\mathbb{C}}

\newcommand{\la}{\mathcal{L}}
\newcommand{\ca}{\mathcal{C}}
\newcommand{\hi}{\mathcal{H}}

\newcommand{\no}[1]{\left\| #1 \right\|}
\renewcommand{\sp}[1]{\left\langle #1 \right\rangle}
% \renewcommand{\sp}[2]{\left\langle #1, #2 \right\rangle}
\renewcommand{\bar}{\overline}

\newcommand*\diff{\mathop{}\!\mathrm{d}}
\newcommand*\rfrac[2]{{}^{#1}\!/_{\!#2}}

\DeclareMathOperator{\argmin}{argmin}
\DeclareMathOperator{\epigraph}{epi}
\DeclareMathOperator{\proximal}{prox}
\DeclareMathOperator{\diagonal}{diag}
\DeclareMathOperator{\domain}{dom}
\DeclareMathOperator{\trace}{tr}

\DeclareMathOperator*{\Argmin}{argmin}
\DeclareMathOperator*{\Min}{min}
\DeclareMathOperator*{\Inf}{inf}
\DeclareMathOperator*{\Sup}{sup}
\DeclareMathOperator*{\Lim}{lim}

\DeclareMathOperator*{\Sum}{\sum}
\DeclareMathOperator*{\Int}{\int}

\renewcommand{\appendixtocname}{Додаток}
\renewcommand{\appendixpagename}{Додаток}
\renewcommand{\appendixname}{Додаток}
\makeatletter
\let\oriAlph\Alph
\let\orialph\alph
\renewcommand{\@resets@pp}{\par
  \@ppsavesec
  \stepcounter{@pps}
  \setcounter{section}{0}%
  \if@chapter@pp
    \setcounter{chapter}{0}%
    \renewcommand\@chapapp{\appendixname}%
    \renewcommand\thechapter{\@Alph\c@chapter}%
  \else
    \setcounter{subsection}{0}%
    \renewcommand\thesection{\@Alph\c@section}%
  \fi
  \if@pphyper
    \if@chapter@pp
      \renewcommand{\theHchapter}{\theH@pps.\oriAlph{chapter}}%
    \else
      \renewcommand{\theHsection}{\theH@pps.\oriAlph{section}}%
    \fi
    \def\Hy@chapapp{appendix}%
  \fi
  \restoreapp
}
\makeatother

\renewcommand\thempfootnote{\alph{mpfootnote}}
\newcommand{\todo}[1]{\footnote{\textcolor{red}{TODO}: #1}}

\newcommand{\cover}[2]{
\begin{center}
\hfill \break \bf
  М{\smallІНІСТЕРСТВО ОСВІТИ ТА НАУКИ} У{\smallКРАЇНИ} \\
  К{\smallИЇВСЬКИЙ НАЦІОНАЛЬНИЙ УНІВЕРСИТЕТ ІМЕНІ} Т{\smallАРАСА} Ш{\smallЕВЧЕНКА} \\ 
  Ф{\smallАКУЛЬТЕТ КОМП'ЮТЕРНИХ НАУК ТА КІБЕРНЕТИКИ} \\
  К{\smallАФЕДРА ОБЧИСЛЮВАЛЬНОЇ МАТЕМАТИКИ}
\end{center}

\vfill 

\begin{center}
  \LARGE \bf
  Звіт до лабораторної роботи №{#1} на тему \\ 
  \guillemotleft{#2}\guillemotright
\end{center}

\vfill 

\begin{flushright}
  \large \bf 
  Виконав студент групи ОМ-3 \\
  
  Скибицький Нікіта
\end{flushright}

\vfill 

\begin{center}
  \large \bf
  Київ --- 2019
\end{center}

\thispagestyle{empty} 
\newpage
}
\newcommand{\people}{
    У виконанні роботи брали участь:
    \begin{itemize}
        \item Основні учасники:
        \begin{itemize}
            \item Скибицький Нікіта
            \item Сергієнко Тетяна
            \item Тихонравова Юлія
            \item Ковальчук Віктор
            \item Кузьмінова Катерина 
            \item Антипова Аліса
        \end{itemize}
        \item Також допомагали:
        \begin{itemize}
            \item Пушкін Денис
            \item Єрмаков Артур
            \item Бельо Андрій
            \item Гронь Ілля
        \end{itemize}
    \end{itemize}
    \newpage
}

\author{Скибицький Нікіта}
\date{\today}

\allowdisplaybreaks
\numberwithin{equation}{section}
\linespread{1.15}

\begin{document}

\coverleader{3}{Мурашиний алгоритм}

\people

\tableofcontents

\section{Постановка задачі}

Розглянемо відому задачу про ранець: є $T$ типів предметів, причому предметів $i$-го типу рівно $q_i$ штук. Предмет $i$-го типу характеризується значеннями своєї корисності $u_i$ та ваги $w_i$. Окрім цього є ранець який вміщує довільну кількість предметів сумарною вагою не більше $W$. Необхідно вибрати підмножину заданих предметів яку можна розмістити у ранці, з максимальною сумарною корисністю. \medskip

Тобто, ставиться задача
\begin{equation}
    U(c) = \Sum_{i = 1}^T c_i \cdot u_i \xrightarrow[c \in \mathcal{C}]{} \max,
\end{equation}
де $\mathcal{C}$ --- допустима область, яка визначається наступним чином:
\begin{equation}
    \mathcal{C} = \left\{ c \in \ZZ^T \middle| \forall i: 0 \le c_i \le q_i \land \Sum_{i = 1}^T c_i \cdot w_i \le W \right\}.
\end{equation}

Зауважимо, що можна також записати
\begin{equation}
    U(c) = \langle c, u \rangle,
\end{equation}
і
\begin{equation}
    \mathcal{C} = \left\{ c \in Q_1 \times Q_2 \times \ldots \times Q_T \middle| \langle c, w \rangle \le W \right\},
\end{equation}
де $Q_i = \ZZ \cap [0, q_i]$.

\section{Опис алгоритму}

Розглянемо популяцію з $N$ мурах, які протягом $M$ ітерацій намагаються спакувати свій ``ранець'' (наприклад, підготувати запаси на зиму, які необхідно розмістити в обмеженому просторі мурашника). Уявімо, що на кожній ітерації кожна мурашка пакує свій уявний ранець ходячи туди-сюди до потрібних їй предметів, залишаючи на своєму шляху феромени, і керуючись у вже наявними із попередніх ітерацій фероменами для вибору шляху.

\subsection{Феромени які залишає мурашка}

А саме, нехай муршака знайшла розв'язок $c$ з сумарною вагою $w$ і сумарною корисністю $u$, тоді кількість фероменів $f_i$ залишених на шляху до купки об'єктів типу $i$ описується функцією $f$:
\begin{equation}
    f_i = f(c_i, w, u, w_i, u_i),
\end{equation}
на яку накладаються наступні умови:
\begin{align}
    \frac{\partial f}{\partial w} &< 0, \\
    \frac{\partial f}{\partial u} &> 0, \\
    \frac{\partial f}{\partial w_i} &< 0, \\
    \frac{\partial f}{\partial u_i} &> 0, \\
    \frac{\partial f}{\partial c_i} &> 0, \\
    f(0, \ldots) &= 0.
\end{align}

Наприклад, 
\begin{equation}
    f(w, u, w_i, u_i) = \ln (c_i + 1) \cdot \frac{u_i}{w_i} \cdot \frac{u}{w}.
\end{equation}

\subsection{Процес побудови розв'язку}

Нехай з попередньої ітерації вже наявні феромени у кількості $f_i$ на шляху до купки предметів $i$-го типу, тоді поки у ранець можна покласти ще хоча б один предмет мурашка обчислює наступні характеристики $\chi_i$ кожного типу предметів який ще може влізти у ранець:
\begin{equation}
    \chi_i = \chi(u_i, w_i, f_i),
\end{equation}
де на функцію $\chi$ накладаються наступні умови
\begin{equation}
    \frac{\partial \chi}{\partial w_i} < 0, \qquad
    \frac{\partial \chi}{\partial u_i} > 0, \qquad
    \frac{\partial \chi}{\partial f_i} > 0.
\end{equation}

Наприклад,
\begin{equation}
    \chi(w_i, u_i, f_i) = \frac{u_i}{w_i} \cdot (1 + \rho \cdot f_i),
\end{equation}
де $\rho \in (0, 2)$ --- певний коефіцієнт, наприклад $\rho = 1$. \medskip

На основі обчислених характеристик обчислюються ймовірності вибору кожного типу предметів:
\begin{equation}
    p_i = \frac{\chi_i}{\Sum_{i = 1}^T \chi_i}.
\end{equation}

\subsection{Процес оновлення фероменів}

Якщо до ітерації $i$ кількість фероменів на шляху до $j$-ої купки предметів дорівнює $F_{i,j}$, а на $i$-ій ітерації додалося $f_{i,j}$, то
\begin{equation}
    F_{i+1,j} = \alpha \cdot f_{i, j} + (1 - \alpha) \cdot F_{i, j},
\end{equation}
де $\alpha \in (0, 1)$ --- певний коефіцієнт, наприклад $\alpha = 0.1$.

\section{Код}

\subsection{Процес побудови розв'язку}

\inputminted{python}{../../code/knapsack/generate_solution.py}

\subsection{Процес обчислення фероменів}

\inputminted{python}{../../code/knapsack/calculate_feroments.py}

\subsection{Програма-драйвер і основний алгоритм}

\inputminted{python}{../../code/knapsack/main.py}

\section{Результати}

Проводилося тестування на відносно складній задачі ($T = 100$) але однорідній задачі. Однорідність означає, що
\begin{equation}
    \frac{\max_i u_i}{\min u_i} \le 2 \quad \land \quad \frac{\max_i w_i}{\min w_i} \le 2
\end{equation}

Початковий результат отриманий випадковим чином складав 80\% теоретичного максимуму:
\begin{verbatim}
it #00: best_u = 11882
\end{verbatim}

За 10 ітерацій алгоритм досяг покращення у 17\% у порівнянні з початковим результатом, тобто досяг результату у 97\% від теоретичного максимуму:
\begin{verbatim}
it #10: best_u = 13942
\end{verbatim}

Опісля відбувається стагнація: за наступні 100 ітерацій прогрес склав ще 2\%, у результаті чого алгоритм підійшов до 99\% теоретичного максимуму.
\begin{verbatim}
it #99: best_u = 14135
\end{verbatim}

\end{document}