\documentclass[a4paper, 12pt]{article}
\usepackage[T2A,T1]{fontenc}
\usepackage[utf8]{inputenc}
\usepackage[english, ukrainian]{babel}
\usepackage{amsmath}
\usepackage{amssymb}
\usepackage{mathtools}
\usepackage{euler}
% \usepackage[dvipsnames]{xcolor}
\usepackage{xcolor}
\usepackage[margin=1.5cm]{geometry}
\usepackage{float}
\usepackage{multirow}
\usepackage{multicol}
\usepackage{url}
\usepackage[unicode=true, colorlinks=true, linktoc=all, linkcolor=blue]{hyperref}
\usepackage{cite}
\usepackage{amsthm}
\usepackage{thmtools}
\usepackage[framemethod=TikZ]{mdframed}
\usepackage[toc,page,title,titletoc]{appendix}
\usepackage{bookmark}
\usepackage[nottoc,notlot,notlof]{tocbibind}
\usepackage{minted}
% \usepackage[hang]{footmisc}
\theoremstyle{definition}
\mdfdefinestyle{mdbluebox}{%
	roundcorner = 10pt,
	linewidth=1pt,
	skipabove=12pt,
	innerbottommargin=9pt,
	skipbelow=2pt,
	nobreak=true,
	linecolor=blue,
	backgroundcolor=TealBlue!5,
}
\declaretheoremstyle[
	headfont=\sffamily\bfseries\color{MidnightBlue},
	mdframed={style=mdbluebox},
	headpunct={\\[3pt]},
	postheadspace={0pt}
]{thmbluebox}

\mdfdefinestyle{mdredbox}{%
	linewidth=0.5pt,
	skipabove=12pt,
	frametitleaboveskip=5pt,
	frametitlebelowskip=0pt,
	skipbelow=2pt,
	frametitlefont=\bfseries,
	innertopmargin=4pt,
	innerbottommargin=8pt,
	nobreak=true,
	linecolor=RawSienna,
	backgroundcolor=Salmon!5,
}
\declaretheoremstyle[
	headfont=\bfseries\color{RawSienna},
	mdframed={style=mdredbox},
	headpunct={\\[3pt]},
	postheadspace={0pt},
]{thmredbox}

\declaretheorem[style=thmbluebox,name=Теорема,numberwithin=section]{theorem}
\declaretheorem[style=thmbluebox,name=Теорема,numbered=no]{theorem*}
\declaretheorem[style=thmbluebox,name=Лема,sibling=theorem]{lemma}
\declaretheorem[style=thmbluebox,name=Лема,numbered=no]{lemma*}
\declaretheorem[style=thmbluebox,name=Твердження,sibling=theorem]{proposition}
\declaretheorem[style=thmbluebox,name=Наслідок,sibling=theorem]{corollary}
\declaretheorem[style=thmredbox,name=Приклад,sibling=theorem]{example}

\mdfdefinestyle{mdgreenbox}{%
	skipabove=8pt,
	linewidth=2pt,
	rightline=false,
	leftline=true,
	topline=false,
	bottomline=false,
	linecolor=ForestGreen,
	backgroundcolor=ForestGreen!5,
}
\declaretheoremstyle[
	headfont=\bfseries\sffamily\color{ForestGreen!70!black},
	bodyfont=\normalfont,
	spaceabove=2pt,
	spacebelow=1pt,
	mdframed={style=mdgreenbox},
	headpunct={ --- },
]{thmgreenbox}

\mdfdefinestyle{mdblackbox}{%
	skipabove=8pt,
	linewidth=3pt,
	rightline=false,
	leftline=true,
	topline=false,
	bottomline=false,
	linecolor=black,
	backgroundcolor=RedViolet!5!gray!5,
}
\declaretheoremstyle[
	headfont=\bfseries,
	bodyfont=\normalfont\small,
	spaceabove=0pt,
	spacebelow=0pt,
	mdframed={style=mdblackbox}
]{thmblackbox}

% \theoremstyle{theorem}
\declaretheorem[name=Запитання,sibling=theorem,style=thmblackbox]{ques}
\declaretheorem[name=Вправа,sibling=theorem,style=thmblackbox]{exercise}
\declaretheorem[name=Зауваження,sibling=theorem,style=thmgreenbox]{remark}
\declaretheorem[name=Припущення,sibling=theorem,style=thmblackbox]{assumption}

\theoremstyle{definition}
\newtheorem{claim}[theorem]{Твердження}
\newtheorem{definition}[theorem]{Визначення}
\newtheorem{fact}[theorem]{Факт}

\newtheorem{problem}{Задача}[section]
\newtheorem{sproblem}[problem]{Задача}
\newtheorem{dproblem}[problem]{Задача}
\renewcommand{\thesproblem}{\theproblem$^{\star}$}
\renewcommand{\thedproblem}{\theproblem$^{\dagger}$}

\makeatletter
\newenvironment{solution}[1][\solutionname]{\par
  \pushQED{\qed}%
  \normalfont \topsep6\p@\@plus6\p@\relax
  \trivlist
%<amsbook|amsproc>  \itemindent\normalparindent
  \item[\hskip\labelsep
%<amsbook|amsproc>        \scshape
%<amsart|amsthm>        \itshape
\itshape 
    #1\@addpunct{.}]\ignorespaces
}{%
  \popQED\endtrivlist\@endpefalse
}
%    \end{macrocode}
%    Default for \cn{proofname}:
%    \begin{macrocode}
\providecommand{\solutionname}{Розв'язок}

\makeatother
\renewcommand{\phi}{\varphi}
\renewcommand{\epsilon}{\varepsilon}

\newcommand{\NN}{\mathbb{N}}
\newcommand{\ZZ}{\mathbb{Z}}
\newcommand{\QQ}{\mathbb{Q}}
\newcommand{\RR}{\mathbb{R}}
\newcommand{\CC}{\mathbb{C}}

\newcommand{\la}{\mathcal{L}}
\newcommand{\ca}{\mathcal{C}}
\newcommand{\hi}{\mathcal{H}}

\newcommand{\no}[1]{\left\| #1 \right\|}
\renewcommand{\sp}[1]{\left\langle #1 \right\rangle}
% \renewcommand{\sp}[2]{\left\langle #1, #2 \right\rangle}
\renewcommand{\bar}{\overline}

\newcommand*\diff{\mathop{}\!\mathrm{d}}
\newcommand*\rfrac[2]{{}^{#1}\!/_{\!#2}}

\DeclareMathOperator{\argmin}{argmin}
\DeclareMathOperator{\epigraph}{epi}
\DeclareMathOperator{\proximal}{prox}
\DeclareMathOperator{\diagonal}{diag}
\DeclareMathOperator{\domain}{dom}
\DeclareMathOperator{\trace}{tr}

\DeclareMathOperator*{\Argmin}{argmin}
\DeclareMathOperator*{\Min}{min}
\DeclareMathOperator*{\Inf}{inf}
\DeclareMathOperator*{\Sup}{sup}
\DeclareMathOperator*{\Lim}{lim}

\DeclareMathOperator*{\Sum}{\sum}
\DeclareMathOperator*{\Int}{\int}

\renewcommand{\appendixtocname}{Додаток}
\renewcommand{\appendixpagename}{Додаток}
\renewcommand{\appendixname}{Додаток}
\makeatletter
\let\oriAlph\Alph
\let\orialph\alph
\renewcommand{\@resets@pp}{\par
  \@ppsavesec
  \stepcounter{@pps}
  \setcounter{section}{0}%
  \if@chapter@pp
    \setcounter{chapter}{0}%
    \renewcommand\@chapapp{\appendixname}%
    \renewcommand\thechapter{\@Alph\c@chapter}%
  \else
    \setcounter{subsection}{0}%
    \renewcommand\thesection{\@Alph\c@section}%
  \fi
  \if@pphyper
    \if@chapter@pp
      \renewcommand{\theHchapter}{\theH@pps.\oriAlph{chapter}}%
    \else
      \renewcommand{\theHsection}{\theH@pps.\oriAlph{section}}%
    \fi
    \def\Hy@chapapp{appendix}%
  \fi
  \restoreapp
}
\makeatother

\renewcommand\thempfootnote{\alph{mpfootnote}}
\newcommand{\todo}[1]{\footnote{\textcolor{red}{TODO}: #1}}

\newcommand{\cover}[2]{
\begin{center}
\hfill \break \bf
  М{\smallІНІСТЕРСТВО ОСВІТИ ТА НАУКИ} У{\smallКРАЇНИ} \\
  К{\smallИЇВСЬКИЙ НАЦІОНАЛЬНИЙ УНІВЕРСИТЕТ ІМЕНІ} Т{\smallАРАСА} Ш{\smallЕВЧЕНКА} \\ 
  Ф{\smallАКУЛЬТЕТ КОМП'ЮТЕРНИХ НАУК ТА КІБЕРНЕТИКИ} \\
  К{\smallАФЕДРА ОБЧИСЛЮВАЛЬНОЇ МАТЕМАТИКИ}
\end{center}

\vfill 

\begin{center}
  \LARGE \bf
  Звіт до лабораторної роботи №{#1} на тему \\ 
  \guillemotleft{#2}\guillemotright
\end{center}

\vfill 

\begin{flushright}
  \large \bf 
  Виконав студент групи ОМ-3 \\
  
  Скибицький Нікіта
\end{flushright}

\vfill 

\begin{center}
  \large \bf
  Київ --- 2019
\end{center}

\thispagestyle{empty} 
\newpage
}

\author{Скибицький Нікіта}
\date{\today}

\allowdisplaybreaks
\numberwithin{equation}{section}
\linespread{1.15}

\begin{document}

\cover{2}{Генетичний алгоритм}

\tableofcontents

\section{Неформальний опис алгоритму}

\subsection{Власне опис}

\subsection{Історія винаходу методу}

\subsection{Сфери застосування}

\subsection{Можливі модифікації}

\section{Формалізований опис алгоритму}

Нехай є певна фітнес-функція $f$: $\mathbb{R}^m \to \mathbb{R}$ і ставиться оптимцізаційна задача
\begin{equation}
    f(x) \xrightarrow[x \in \mathcal{C}]{} \min,
\end{equation}
де $\mathcal{C} \subset \mathbb{R}^m$ --- опукла і замкнена допустима множина, наприклад $\mathcal{C} = [x_{\text{min}}, x_{\text{max}}]^m$. \medskip

Розглянемо популяцію з $n$ особин які протягом $M$ поколінь розв'язують цю оптимізаційну задачу (пристосовуються під задану фітнес-функцію). Кожну особину будемо описувати двійковим вектором достатньо довгим для того щоб кодувати десяткові значення аргументів функції $x_i$ з потрібною точністю $\varepsilon$. \medskip

Перше покоління не має досвіду предків і генерується випадковим чином, рівномірно на $\mathcal{C}$. \medskip

Далі з кожним поколінням відбуваються наступні речі:
\begin{itemize}
    \item воно кодується з десяткового представлення у двійкове;
    \item у ньому із заданою ймовірністю $p$ відбувають бітові мутації;
    \item покоління розбивається на пари, між якими відбувається кроссовер (обмін генами);
    \item на основі покоління генерується нове таким чином що найкращі особини мають більшу ймовірність мати нащадка.
\end{itemize}

\section{Код програмного продукту}

\subsection{GenerationDec}

\subsubsection*{Призначення}

Ця процедура заповнює матрицю із заданою кількістю рядків і стовпчиків випадковими десятковими числами із заданого діапазону. \medskip

Діапазон значень фіксований для усіх елементів кожного стовпчика матриці. \medskip

Елементи останнього стовпчика матриці не заповнюються випадковими числами і призначені для обчислення значень фітнес-функції, аргументами якої є всі попередні елементи рядка матриці.

\subsubsection*{Вхідні параметри}

\begin{itemize}
    \item $N, M$ --- цілі невід'ємні числа
    \item $X_{\text{min}}(1..M), X_{\text{max}}(1..M)$ --- масиви дійсних чисел, $X_{\text{min}}[i] < X_{\text{max}}[i]$, $i = \overline{1..M}$;
\end{itemize}

\subsubsection*{Вихідні параметри}

\begin{itemize}
    \item $G(1..N, 1..M+1)$ --- матриця випадкових значень.
\end{itemize}

\subsubsection*{Обчислення}

Заповнює елементи матриці $G[i,j]$, $i = \overline{1..N}$, $j = \overline{1..M}$, $X_{\text{min}}[j] \le G[i, j] \le X_{\text{max}}[j]$.

\subsubsection*{Вказівки}

Для генерування випадкових чисел з заданого діапазону використовувати функцію Generate пакета Random Tools.

\subsubsection*{Власне реалізація}

\inputminted[firstline=7, lastline=40]{python}{../code/generation_dec.py}

\subsection{Mutation}

\subsubsection*{Призначення}

Опрацьовує в циклі кожен елемент прямокутної матриці заданого розміру з елементами 0 або 1 за наступним правилом: 
\begin{itemize}
    \item якщо згенероване випадкове число менше заданого порогового значення, то відповідний елемент матриці інвертується (0 в 1 або 1 в 0);
    \item інакше елемент матриці не змінюється. 
\end{itemize}

Результат зберігається у новій матриці.

\subsubsection*{Вхідні параметри}

\begin{itemize}
    \item $G$ --- прямокутна матриця зі значеннями 0 або 1;
    \item $p$ --- дійсне число $0 < p \ll 1$ --- ймовірність мутації.
\end{itemize}

\subsubsection*{Вихідні параметри}

\begin{itemize}
    \item $G_{\text{mut}}$ --- прямокутна матриця зі значеннями 0 або 1, розмірність якої дорівнює розмірності $G$;
    \item $S_{\text{mut}}$ --- лічильних загальної кількості мутацій, що були виконані для матриці $G$.
\end{itemize}

\subsubsection*{Обчислення}

\begin{itemize}
    \item В циклі опрацювати кожен елемент матриці $G$, перевіряючи умову, що згенероване випадкове число менше заданого параметра $p$. 
    \item При виконанні умови виконати мутацію, інакше елемент матриці не змінюється. 
    \item При кожній мутації збільшувати лічильник на одиницю.
\end{itemize}

\subsubsection*{Вказівки}

Для генерування випадкових чисел з заданого діапазону можна використовувати функцію Generate з пакету Random Tools.

\subsubsection*{Власне реалізація}

\inputminted[firstline=6, lastline=15]{python}{../code/mutation.py}

\subsection{Crossover}

\subsubsection*{Призначення}

Процедура опрацьовує вхідну прямокутну матрицю з парною $2N$ кількістю рядків і довільною кількістю стовпчиків, елементами якої є числа 0 або 1. \medskip

Рядки матриці об'єднуються в пари, пари утворюються за значенням номерів рядків матриці, вказаними у двох списка розмірності $N$. \medskip

Кожна пара рядків матриці опрацьовується наступним чином:
\begin{itemize}
    \item Генерується випадкове ціле число $1 \le \text{cros} \le L$ --- кількість стовпчиків матриці. 
    \item Пара рядків матриці, яка зараз опрацьовується, обмінюється між собою елементами стовпчиків з номерами від $1$ до $\text{cros}$. 
    \item Результати записуються у нову матрицю, розмірність якої дорівнює розмірності вхідної матриці.
\end{itemize}

\subsubsection*{Вхідні параметри}

\begin{itemize}
    \item $N$ --- ціле додатнє число;
    \item $G(1..2N,1..L)$ --- матриця з елементами 0 або 1;
    \item $M_{\text{list}}(1..N), F_{\text{list}}(1..N)$ --- списки цілих чисел зі значеннями в інтервалі від $1$ до $2N$.
\end{itemize}

\subsubsection*{Вихідні параметри}

\begin{itemize}
    \item Матриця $G_{\text{cros}}(1..2N,1..L)$ з елементами 0 або 1 як результат процедури кроссовера (схрещування).
\end{itemize}

\subsubsection*{Обчислення}

\begin{itemize}
    \item Організується цикл по елементам списків $M_{\text{list}}[i], F_{\text{list}}[i]$, $i = \overline{1..N}$. 
    \item Послідовно опрацьовується кожна пара рядків $G[M_{\text{list}}[i],1..L]$ и $G[F_{\text{list}}[i],1..L]$. 
    \item Для кожної такої пари рядків генерується випадкове число $\text{cros}$ від $1$ до $L$. 
    \item Ці рядки обмінюються між собою елементами від $1$ до $\text{cros}$. 
    \item Результат змінених рядків записується у нову матрицю $G_{\text{cros}}(1..2N,1..L)$.
\end{itemize}

\subsubsection*{Вказівки}

\begin{itemize}
    \item Для генерування випадкових чисел з заданого діапазону можна використовувати функцію Generate з пакету Random Tools.
    \item Для обчислення кількості рядків або стовпчиків матриці можна використовувати функції RowDimension і ColumnDimension з пакету LinearAlgebra.
\end{itemize}

\subsubsection*{Власне реалізація}

\inputminted[firstline=5, lastline=30]{python}{../code/crossover.py}

\subsection{Parents}

\subsubsection*{Призначення}

послідовність натуральних чисел від $1$ до $2N$ розбивається на два списки натуральних чисел таким чином, що елементи кожгого списку вибираються випадковим чином з діапазону $[1..2N]$ з умовою, що елементи списків з одинаковими номерами різні.

\subsubsection*{Вхідні параметри}

\begin{itemize}
    \item $N$ --- кількість елементів списків.
\end{itemize}

\subsubsection*{Вихідні параметри}

\begin{itemize}
    \item Списки $m_{\text{list}}(1..N), f_{\text{list}}(1..N)$.
\end{itemize}

\subsubsection*{Обчислення}

\begin{itemize}
    \item Організується циклічний процес по всім елементам списку. 
    \item Елементам першого списку значення присвоють випадковим чином, а елементам другого списку випадковим чином, але з умовою, що $m_{\text{list}}[i] \ne f_{\text{list}}[i]$.
    \item При рівності робиться ще спроба присвоєння випадково числа до виконання умови.
\end{itemize}

\subsubsection*{Вказівки}

Для генерування випадкових чисел з заданого діапазону можна використовувати функцію Generate з пакету Random Tools.

\subsubsection*{Власне реалізація}

\inputminted[firstline=6, lastline=12]{python}{../code/parents.py}

\subsection{BinDecParam}

\subsubsection*{Призначення}

Це допоміжна процедура для CodBinary і CodDecimal, обчислює параметри для прямого і зворотнього перетворення із заданою точністю із заданого інтервалу дійсних чисел у бінарну послідовність. 

\subsubsection*{Вхідні параметри}

\begin{itemize}
    \item $M$ --- розмірності масивів.
    \item $X_{\text{min}}(1..M)$ --- масив, де $X_{\text{min}}[j]$ --- мінімальне значення діапазону $j$;
    \item $X_{\text{max}}(1..M)$ --- масив, де $X_{\text{max}}[j]$ --- максимальне значення діапазону $j$;
    \item $\varepsilon$ --- точність представлення десяткових чисел двійковим кодом.
\end{itemize}

\subsubsection*{Вихідні параметри}

\begin{itemize}
    \item $nn[i]$, $i = \overline{1..M}$ --- список цілих чисел, містить значення кількості бінарних розрядів які необхідні для кодування довільного дійсного числа з діапазону $[X_{\text{min}}[i], X_{\text{max}}[i]]$ з точністю $\varepsilon$.
    \item $dd[i]$, $i = \overline{1..M}$ --- дійсне число, містить значення дискретності для представлення дійсного числа із заданого діапазону цілим числом.
    \item $NN[j + 1] = \sum_{i = 1}^j nn[i]$, $j = \overline{1..M}$, $NN[1] = 0$ --- список цілих чисел.
\end{itemize}

\subsubsection*{Вказівки}

Всі вихідні параметри описуються з атрибутом global.

\subsubsection*{Обчислення}

\begin{align}
    nn[i] &= \left[ \log_2 \left( \frac{X_{\text{max}}[i] - X_{\text{min}}[i]}{\varepsilon} \right) \right] + 1, \quad i = \overline{1..M}, \\
    dd[i] &= \frac{X_{\text{max}}[i] - X_{\text{min}}[i]}{2^{nn[i]}} \le \varepsilon, \quad i = \overline{1..M}, \\
    NN[j + 1] &= \sum_{i = 1}^j nn[i], \quad j = \overline{1..M}, \quad NN[1] = 0.
\end{align}

\subsubsection*{Власне реалізація}

\inputminted[firstline=9, lastline=51]{python}{../code/bin_dec_param.py}

\subsection{CodBinary}

\subsubsection*{Призначення}

Процедура виконує кодування довільного дійсного числа $x_{\text{dec}}$ з заданого діапазону $[x_{\text{min}}..x_{\text{max}}]$ з заданою точністю $\varepsilon$ у послідовність з 0 і 1 фіксованої довжини. \medskip

Процедура працює у парі з процедурою CodDecimal, яка виконує зворотнє перетворення. \medskip

Допоміжні параметри обчислюються процедурою BinDecParam.

\subsubsection*{Вхідні параметри}

\begin{itemize}
    \item $x_{\text{dec}}$ --- десяткове число;
    \item $x_{\text{min}}$ --- мінімальне значення числа, що кодується;
    \item $l$ --- ціле число, максимальна кількість двійкових розрядів для представлення довільного числа із заданого діапазону із заданою точністю;
    \item $d$ --- дискретність кодування дійсного числа $x_{\text{dec}}$ цілим числом.
\end{itemize}

\subsubsection*{Вихідні параметри}

\begin{itemize}
    \item $X_{\text{bin}}$ --- список з $l$ розрядів двійкового числа, молодші розряді йдуть спочатку. 
\end{itemize}

У разі потреби старші розряди дозаповнюються нулями.

\subsubsection*{Обчислення}

Ціле число частин величини $d$ для заданого числа $x_{\text{dec}}$ можно обчислити як
\begin{equation}
    xx = \left[ \frac{x_{\text{dec}} - x_{\text{min}}}{d} \right].
\end{equation}

Ціле число $xx$ записуємо у двійковій формі і доповнюємо старші розряди нулями, якщо їхня кількість менше $l$.

\subsubsection*{Вказівки}

Значення $l$ і $d$ обчислюються процедурою BinDecParam і не можуть задаватися довільно. \medskip

Перетворення цілого десяткового числа у двійковий код (список 0 і 1) можна виконати функцією convert(xx, base, 2).

\subsubsection*{Власне реалізація}

\inputminted[firstline=7, lastline=28]{python}{../code/cod_binary.py}

\subsection{CodDecimal}

\subsubsection*{Призначення}

Процедура виконує перетворення закодованого послідовністю з 0 і 1 двійкового представлення числа з заданого діапазону $[x_{\text{min}}..x_{\text{max}}]$ із заданою точністю $\varepsilon$ у його звичайне десяткове представлення. \medskip

Процедура працює в парі з процедурою CodBinary, яка виконує зворотнє перетворення. 

\subsubsection*{Вхідні параметри}

\begin{itemize}
    \item $x_{\text{bin}}$ --- послідовність розрядів двійкового числа;
    \item $x_{\text{min}}$ --- мінімальне значення десяткового числа;
    \item $d$ --- дискретність кодування дійсного чиисла $x_{\text{dec}}$ цілим числом.
\end{itemize}

\subsubsection*{Вихідні параметри}

\begin{itemize}
    \item $x_{\text{dec}}$ --- десяткове число. 
\end{itemize}

\subsubsection*{Обчислення}

\begin{itemize}
    \item Послідовність розрядів $x_{\text{bin}}$ перетворюється у ціле число $x_{\text{dec}}'$. 
    \item Далі за допомогою цілого числа $d$ відновлюємо десяткове число $x_{\text{dec}} = x_{\text{min}} + d \cdot x_{\text{dec}}'$.
\end{itemize}

\subsubsection*{Вказівки}

Значення $l$ і $d$ обчислюються процедурою BinDecParam і не можуть задаватися довільно. \medskip

Перетворення двійкового коду (список 0 і 1) у послідовність десяткових розрядів цілого десяткового числа можна виконати функцією convert(xbin, base, 2, 10).

\subsubsection*{Власне реалізація}

\inputminted[firstline=9, lastline=24]{python}{../code/cod_decimal.py}

\subsection{ACodBinary}

\subsubsection*{Призначення}

Послідовно перетворює дійсні числа, а саме елементи матриці $G_{\text{dec}}(1..N,1..M+1)$, яка складається з $N$ рядків і $M + 1$ стовпчиків у двійковий код. \medskip

Перетворення виконуються над елементами тільки $M$ перших стовпчиків. \medskip

Для перетворення кожного елемента матриці використовуються процедури CodBinary і BinDecParam. \medskip

Результати перетворення дійсних чисел зберігаються у матрицю Gbin з елементами 0 або 1. \medskip

Кожному стовпчику матриці $G_{\text{dec}}$ відповідає фіксована кількість стовпчиків матриці $G_{\text{bin}}$, яка визначається діапазоном дійсних значень які кодуються і точністю їхнього представлення (див. процедуру CodBinary, параметр $l$).

\subsubsection*{Вхідні параметри}

\begin{itemize}
    \item $N, M$ --- розмірності матриці $G_{\text{dec}}(1..N,1..M+1)$;
    \item Матриця $G_{\text{dec}}(1..N,1..M+1)$;
    \item $X_{\text{min}}(1..M)$ --- масив, де $X_{\text{min}}[j]$ --- мінімальне значення елементів стовпчика $j$;
    \item Глобальні параметри процедури BinDecParam: $nn, dd, NN$.
\end{itemize}

\subsubsection*{Вихідні параметри}

\begin{itemize}
    \item Матриця $G_{\text{bin}}$ з елементами 0 або 1, яка складається з $N$ рядків, кількість стовпчиків матриці визначається значенням $NN \cdot (M + 1)$.
\end{itemize}

\subsubsection*{Обчислення}

В циклі опрацьовуємо кожний елемент матриці $G_{\text{dec}}$, використовуємо процедуру CodBinary, записуємо результати перетворення у матрицю $G_{\text{bin}}$ розмірності $(1..N, 1..NN \cdot [M+1])$.

\subsubsection*{Вказівки}

Використовувати для обробки кожного елемента вихідної матриці процедуру CodBinary.

\subsubsection*{Власне реалізація}

\inputminted[firstline=31, lastline=73]{python}{../code/cod_binary.py}

\subsection{ACodDecimal}

\subsubsection*{Призначення}

Опрацьовує вміст матриці $G_{\text{bin}}$ з елементами 0 або 1 використовуючи структуру матриці і перетворює послідовності з 0 і 1 заданих довжин у дійсні числа.

\subsubsection*{Вхідні параметри}

\begin{itemize}
    \item $N, M$ --- розмірності матриці $G_{\text{dec}}(1..N,1..M+1)$;
    \item Матриця $G_{\text{bin}}$;
    \item $X_{\text{min}}(1..M)$ --- масив, де $X_{\text{min}}[j]$ --- мінімальное значення елемениів стовпчика $j$;
    \item Глобальні параметри $nn, dd, NN$ процедури BinDecParam.
\end{itemize}

\subsubsection*{Вихідні параметри}

\begin{itemize}
    \item Матриця $G_{\text{dec}}(1..N,1..M+1)$.
\end{itemize}

\subsubsection*{Обчислення}

\begin{itemize}
    \item У циклі послідовно опрацьовуються всі рядки матриці $G_{\text{bin}}$. 
    \item З кожного рядка вибираються $M$ послідовностей з 0 і 1, які опрацьовує процедура CodDecimal. 
    \item Отримані дійсні числа записуються у матрицю $G_{\text{dec}}(1..N,1..M+1)$.
\end{itemize}

\subsubsection*{Власне реалізація}

\inputminted[firstline=27, lastline=66]{python}{../code/cod_decimal.py}

\subsection{Adapt}

\subsubsection*{Призначення}

Генерирує масив $N$ цілих невід'ємних чисел, сума яких дорівнює $N$. \medskip

Кожне значення з номером $j$ масиву обчислюється як кількість потраплянь випадкового числа рівномірно розподіленого на відрізку $[0..1]$ у підінтервал з номером $1 \le j \le N$, що належить відрізну $[0..1]$. \medskip

Сума довжин всіх підінтервалів дорівнює одиниці.

\subsubsection*{Вхідні параметри}

\begin{itemize}
    \item $p$ --- послідовність $N$ дійсних чисел;
    \item $N$ --- ціле додатнє число.
\end{itemize}

\subsubsection*{Вихідні параметри}

\begin{itemize}
    \item $\text{num}$ --- масив $N$ цілих невід'ємних чисел, сума яких дорівнює $N$.
\end{itemize}

\subsubsection*{Обчислення}

\begin{itemize}
    \item Перетворюємо послідовнітсь $p$:
    \begin{itemize}
        \item Знаходимо мінимальний елемент $p$.
        \item Віднімаємо від кожного елемента $p$ мінімальний елемент, утворюємо $p_1$.
    \end{itemize}
    \item Знаходимо суму всіх елементів $p_1$;
    \item Ділимо кожен елемент $p_1$ на суму значень його елементів, утворюємо $p_2$.
    \item Утворюємо список префіксних сум $p_2$:
	\begin{equation}
	    \text{adaptability} = [0, p_2[1], p_2[1] + p_2[2], \ldots, p_2[1] + p_2[2] + \ldots + p_2[N]].
	\end{equation}
    \item $N$ разів генеруємо випадкове число $0 \le \text{roll} < 1$ і перевіряємо умову належності випадкового числа $\text{roll}$ одному з діапазонів списку $\text{adaptability}$:
    \begin{itemize}
        \item Якщо $\text{adaptability}[j] \le \text{roll} < \text{adaptability}[j + 1]$, то $\text{num}[j] = \text{num}[j] + 1$.
    \end{itemize}
\end{itemize}

\subsubsection*{Вказівки}

Для генерування випадкових чисел з заданого діапазону можна використовувати функцію Generate з пакету Random Tools.

\subsubsection*{Власне реалізація}

\inputminted[firstline=6, lastline=31]{python}{../code/adapt.py}

\subsection{Best і Worst}

\subsubsection*{Призначення}

Процедури опрацьовують $(M+1)$-ий стовпчик матриці $G_{\text{dec}}$, який містить дійсні значення деякої функції від $M$ аргументів, самі аргументи містять у стовпчиках з $1$ по $M$.

\subsubsection*{Обчислення}

Процедура Best обчислює номер рядка з мінімальним значенням функції, а процедура Worst --- номер рядка з максимальним значенням функції. 

\subsubsection*{Вхідні параметри}

\begin{itemize}
    \item $G_{\text{dec}}$ --- вищезгадана матриця.
\end{itemize}

\subsubsection*{Вихідні параметри}

\begin{itemize}
    \item Кожна з процедур повертає знайдене ціле число.
\end{itemize}

\subsubsection*{Власне реалізація}

\inputminted[firstline=5, lastline=36]{python}{../code/best_worst.py}

\subsection{NewGeneration}

\subsubsection*{Призначення}

Опрацьовує рядки вхідної матриці виходячи зі значень вхідного цілочисельного масиву з невід'ємними елементами (сума елементов цілочисльного масиву дорівнює кількості рядків матриці). \medskip

Розмірність масива і кількість рядків матриці рівні. \medskip

Рядок з номером $j$ вхідної матриці записується у вихідну матрицю таку кількість разів, яка дорівнює цілому значенню з номером $j$ масива. \medskip

Якщо на позиції $j$ знаходиться число $0$, то рядок у вихідну матрицю не записується.

\subsubsection*{Вхідні параметри}

\begin{itemize}
    \item $G_{\text{dec}}(1..N,1..M+1)$ --- матриця;
    \item $\text{num}(1..N)$ --- масив цілочисльних значень.
\end{itemize}

\subsubsection*{Вихідні параметри}

\begin{itemize}
    \item $G_{\text{dec}_{\text{new}}}(1..N,1..M+1)$ --- вихідна матриця.
\end{itemize}

\subsubsection*{Власне реалізація}

\inputminted[firstline=5, lastline=11]{python}{../code/new_generation.py}

\subsection{Програма-драйвер}

\inputminted[lastline=43]{python}{../code/main.py}

\section{Тестування програмного продукту}

% \newpage
% \bibliography{main}
% \bibliographystyle{ieeetr}

\end{document}